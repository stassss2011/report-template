\section*{Постановка задачи}
Необходимо сделать нормальный шаблон для отчётов в Политехе. Структура отчётов может быть разной, зависит от требования преподавателя, поэтому файл content.tex отдельно выделен от всех других в шаблоне и не делится на подчасти.
\addcontentsline{toc}{section}{Постановка задачи}

\section{Программа работы}

\section{Теоретическая информация}

\section{Ход выполнения работы}

\subsection{Список}

\begin{itemize}
\item первый элемент списка
\item второй элемент списка
\end{itemize}


\subsection{Картинка}

\begin{figure}[H]
	\begin{center}
		\includegraphics[scale=0.7]{sample}
		\caption{название картинки} 
		\label{pic:pic_name} % название для ссылок внутри кода
	\end{center}
\end{figure}

Текст без отступа (следует за вставкой)

Новый параграф

\noindent Новый параграф с принудительно выключенным отступом


\subsection{Таблица}

\begin{table}[H]
	\caption{ Название таблицы}
	\begin{center}
		\begin{tabular}{|l|l|}
			\hline
			top left & top right\\ \hline
			bot left & bot right\\ \hline
		\end{tabular}
		\label{tabular:tab_examp}
	\end{center}
\end{table}

\begin{landscape}
\subsection{Поворот страницы}
Поворачиваем страницу, потому что можем.
\begin{figure}[H]
    \centering
    \includegraphics[width=26.5cm]{diagram}
    \caption{Да.}
\end{figure}
\end{landscape}

\subsection{Листинг}
\begin{code}
    \inputminted[breaklines=true, xleftmargin=1em,linenos, frame=single, framesep=10pt, firstline=1, lastline=33]{haskell}{listings/Parser.hs}
    \caption{Parser.hs --- функциональный код в массы!}
\end{code}

\section*{Заключение}
\LaTeX\ удобен для создания отчётов, так как сам следит за нумерацией таблиц, рисунков, листингов и отсылок к ним (так, например, здесь всегда будет указан номер рисунка "sample" не зависимо от того, какой он (1,2 или другой) - это рисунок \ref{pic:pic_name}). Не менее важно что весь документ оформлен в едином стиле, а исходные материалы подключаются к отчёту, а не хранятся в нём. Всё это позволяет легко получить качественный отчёт без дополнительных трат на его офрмление.

Исключения, пожалуй, составляют таблицы, так как их значительно сложнее создавать кодом, нежели в графическом редакторе. Но здесь никто не запрещает использовать визуальные средства создания таблиц для \LaTeX\ .
\addcontentsline{toc}{section}{Заключение}